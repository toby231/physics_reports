\documentclass{article}
\usepackage{graphicx} % Required for inserting images
\usepackage{booktabs}
\usepackage{tabu}
\usepackage[margin=1in]{geometry}

\setlength{\parskip}{1em}
\setlength{\parindent}{0em}

\title{GYRO REPORT}
\author{Toby Nguyen - z5416116}
\date{May 2024}

\begin{document}
\begin{center}
    \huge{Gyro Report} \\[10pt]
    \large{Toby Nguyen - z5416116}
\end{center}

\section*{Introduction}
In rotational mechanics, there are phenomena analog to translational mechanics
such as momentum of inertia, torque and angular velocity, acceleration and
momentum. However, certain phenomena in rotational mechanics such as precession
and nutation do not have such analogs and are thus inherently unintuitive. To 
explore these mechanics and to build the intuition, a mathematical derivation is required. 

\subsection*{Precession}
Precession is the motion where the spinning object rotations about a vertical axis, 
in addition to the rotation about its own axis.

In its inertial reference frame, the total torque of a spinning object is related to its angular momentum,
\begin{equation}
    \vec{\tau} = \frac{d\vec{L}}{dt}.
\end{equation}

However from a fixed reference frame, there must be a second term that accounts 
for the rotation of the inertial reference frame, i.e
\begin{equation}
    \vec{\tau} = \frac{d\vec{L}}{dt} + \vec{\omega} \times \vec{L}.
\end{equation}

This second term suggests that there is a torque that acts perpendicular to
the angular momentum and velocity of the spiining object. 
\begin{equation}
    \vec{\tau}_p = \vec{\omega}_s \times \vec{L}_s.
\end{equation}

\subsection*{Nutation}
Nutation refers to the oscillating motion of a spinning object's own axis with respect
to its original axis' position. Nutation occurs when a transient force acts on the 
spinning object in any direction perpendicular to its angular momentum. Hence,
\begin{equation}
    \vec{\tau}_n = k_1\vec{F}_{applied} \times k_2\vec{L}_s. 
\end{equation}

\section*{Aim}
\section*{Method}
\section*{Results}
\section*{Analysis}
\section*{Discussion}
\section*{Conclusion}
\end{document}

\documentclass{article}
\usepackage{graphicx} % Required for inserting images
\usepackage{pgf}
\usepackage{lmodern}
\usepackage{import}
\usepackage{booktabs}
\usepackage{tabu}
\usepackage{float}
\usepackage[hidelinks]{hyperref}
\usepackage{amsmath}
\usepackage{amsfonts}
\usepackage[margin=1in]{geometry}
\usepackage{pythonhighlight}
\usepackage[toc]{appendix}
\usepackage{placeins}

\setlength{\parskip}{1em}
\setlength{\parindent}{0em}

\begin{document}

\section*{Graphs}
Note that any plot using age has to use the Vanderberg 
table which is constrained to 55 globular clusters. Any 
other plot without age will be using the Harris catalogue 
so there are many more clusters plotted.

\subsection*{Stellar populations}
\begin{figure} [H]
    \centering
    \scalebox{0.75}{\input{agefeh.pgf}}
    \caption{Note the density of the dots. We expect 
    metallicities of clusters to be similar at a fixed 
    age. We find that at older ages, there is a greater 
    range in metallicities, indicating potential accreted
    metal poor stars.}
\end{figure}

\begin{figure} [H]
    \centering
    \scalebox{0.75}{\input{agergc.pgf}}
    \caption{We find that there is little to no correlation
    between age of globular clusters and their distance from 
    galactic center. Strangely enough, there are some old 
    globular clusters that are relatively close to the galactic
    center, could potentially indicate accreted or clusters
    moving closer towards the center?}
\end{figure}

\begin{figure} [H]
    \centering
    \scalebox{0.75}{\input{fehrgc.pgf}}
    \caption{Similar to previous plot, there seems to be 
    no correlation between galactic radius and metallicity.
    There is a large variety of globular clusters near 
    each other. Why?}
\end{figure}

\begin{figure} [H]
    \centering
    \scalebox{0.75}{\input{xzfeh.pgf}}
    \caption{Edge on view of the distribution of metallicity.
    Zoomed out doesn't show too much. Note the really metal 
    poor clusters far away from the cluster cluster.}
\end{figure}

\begin{figure} [H]
    \centering
    \scalebox{0.75}{\input{xzfeh1.pgf}}
    \caption{A zoomed in version of the previous plot. We 
    can better see the group of clusters with various 
    metallicities. Strong evidence for accreted globular
    clusters.}
\end{figure}

\subsection*{Stellar kinematics}

\begin{figure} [H]
    \centering
    \scalebox{0.75}{\input{rgcsigv.pgf}}
    \caption{Large range of dispersion velocity at a fixed
    radius, indicating large variety of low mass and high 
    mass stars (different hot and cold orbits).}
\end{figure}

\begin{figure} [H]
    \centering
    \scalebox{0.75}{\input{rgcvr.pgf}}
    \caption{Largely symmetric distribution of rotational 
    velocity, showing lack of asymmetric disturbance in 
    the galaxy. Dotted grey lines displaying $\mu = -4.7$
    and $1\sigma$, $2\sigma$, $3\sigma$.}
\end{figure}

\begin{figure} [H]
    \centering
    \scalebox{0.75}{\input{rgczsigv.pgf}}
    \caption{Edge on view of galaxy, measuring its 
    dispersion velocity. Note that it is predominately
    low dispersion, indicating more spirals/low mass. 
    Note the yellow points in the group, could indicate
    accreted cluster.}
\end{figure}

\begin{figure} [H]
    \centering
    \scalebox{0.75}{\input{rgczvr.pgf}}
    \caption{Similar graph as before, note the purples
    and yellows near the big group.}
\end{figure}

\begin{figure} [H]
    \centering
    \scalebox{0.75}{\input{xysigv.pgf}}
    \caption{Bird eye view of the plots before.}
\end{figure}

\begin{figure} [H]
    \centering
    \scalebox{0.75}{\input{xyvr.pgf}}
    \caption{Bird eye view of the plots before.}
\end{figure}

\begin{figure} [H]
    \centering
    \scalebox{0.75}{\input{yvr.pgf}}
    \caption{Potential counter-rotators? If Y and 
    rotational velocity are opposite signs then 
    possibly accreted.}
\end{figure}

\begin{figure} [H]
    \centering
    \scalebox{0.75}{\input{norm1.pgf}}
    \caption{Radial velocity distribution.}
\end{figure}

\begin{figure} [H]
    \centering
    \scalebox{0.75}{\input{norm2.pgf}}
    \caption{Dispersion velocity distribution.}
\end{figure}

\begin{figure} [H]
    \centering
    \scalebox{0.75}{\input{rgcsigv1.pgf}}
    \caption{Large range of dispersion velocity at a fixed
    radius, indicating large variety of low mass and high 
    mass stars (different hot and cold orbits).}
\end{figure}

\begin{figure} [H]
    \centering
    \scalebox{0.75}{\input{rgcvr1.pgf}}
    \caption{Largely symmetric distribution of rotational 
    velocity, showing lack of asymmetric disturbance in 
    the galaxy. Dotted grey lines displaying $\mu = -4.7$
    and $1\sigma$, $2\sigma$, $3\sigma$.}
\end{figure}

\begin{python}

#Ensuring naming conventions for GCs match
vbt.rename(columns={'#NGC':'ID'},inplace=True)
def add_NGC(s):
    return "NGC " + s
vbt['ID'] = vbt['ID'].apply(add_NGC)

#Dropping irrelevant columns in the dataset
hp1 = hp1.drop(columns=["Name","RA","DEC","L","B","R_Sun"])
hp2 = hp2.drop(columns=["E(B-V)","V_HB","(m-M)V","V_t","M_V,t","U-B","B-V","V-R","spt","ellip"])
hp3 = hp3.drop(columns=["v_LSR","c","r_c","r_h","mu_V","rho_0","lg_tc","lg_th"])
vbt = vbt.drop(columns=["Name","Method", "Figs", "Range","HBtype","R_G","M_V","v_e0","log_sigma_0"])

#Ensuring values are represented in the same way
hp1['ID'] = hp1['ID'].str.replace('"', '')
hp2['ID'] = hp2['ID'].str.replace('"', '')
hp3['ID'] = hp3['ID'].str.replace('"', '')
vbt['ID'] = vbt['ID'].str.replace('"', '')

hp1['ID'] = hp1['ID'].str.strip()
hp2['ID'] = hp2['ID'].str.strip()
hp3['ID'] = hp3['ID'].str.strip()
vbt['ID'] = vbt['ID'].str.strip()

#Merging into two dataframes, one with age and one without age
df0 = pd.merge(hp1,hp2,on='ID')
dfna = pd.merge(df0,hp3,on='ID')
dfa = pd.merge(dfna,vbt,on='ID')
\end{python}

\newpage

\begin{python}
plt.scatter(
    dfna['R_gc'],
    dfna['sig_v'],
    marker='x',
    c=dfna['[Fe/H]']
)

plt.xlim(0,30)
plt.xlabel("Galactic Radius (kpc)")
cbar = plt.colorbar()
cbar.set_label('[Fe/H]')
plt.ylabel("Dispersion velocity (km/s)")
plt.savefig("rgcsigv1.pgf", bbox_inches='tight', pad_inches=0.1)
\end{python}

\begin{python}
# 1. Remove NaN values
dfna_clean = dfna['sig_v'].dropna()

# 2. Plot the histogram to visualize the distribution
plt.hist(dfna_clean, bins=20, density=True, alpha=0.6, color='b', label='Binned Data')

# 3. Fit a normal distribution and overlay the curve (Optional)
mu, std = norm.fit(dfna_clean)  # Get mean and standard deviation
xmin, xmax = plt.xlim()
x = np.linspace(xmin, xmax, 100)
p = norm.pdf(x, mu, std)
plt.plot(x, p, 'k', linewidth=2, label='Normal Fit',linestyle='--')
print(std)

# 4. Add labels and show the plot
plt.xlabel('Velocity Dispersion (km/s)')
plt.ylabel('Density')
plt.title('Binned Data and Normal Distribution Fit')
plt.legend()
plt.savefig("norm2.pgf", bbox_inches='tight', pad_inches=0.1)
plt.show()
\end{python}
\end{document}